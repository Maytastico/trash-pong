% Options for packages loaded elsewhere
\PassOptionsToPackage{unicode}{hyperref}
\PassOptionsToPackage{hyphens}{url}
%
\documentclass[
]{article}
\usepackage{amsmath,amssymb}
\usepackage{lmodern}
\usepackage{iftex}
\usepackage{graphicx}
\usepackage{float}
\ifPDFTeX
  \usepackage[T1]{fontenc}
  \usepackage[utf8]{inputenc}
  \usepackage{textcomp} % provide euro and other symbols
\else % if luatex or xetex
  \usepackage{unicode-math}
  \defaultfontfeatures{Scale=MatchLowercase}
  \defaultfontfeatures[\rmfamily]{Ligatures=TeX,Scale=1}
\fi
% Use upquote if available, for straight quotes in verbatim environments
\IfFileExists{upquote.sty}{\usepackage{upquote}}{}
\IfFileExists{microtype.sty}{% use microtype if available
  \usepackage[]{microtype}
  \UseMicrotypeSet[protrusion]{basicmath} % disable protrusion for tt fonts
}{}
\makeatletter
\@ifundefined{KOMAClassName}{% if non-KOMA class
  \IfFileExists{parskip.sty}{%
    \usepackage{parskip}
  }{% else
    \setlength{\parindent}{0pt}
    \setlength{\parskip}{6pt plus 2pt minus 1pt}}
}{% if KOMA class
  \KOMAoptions{parskip=half}}
\makeatother
\usepackage{xcolor}
\IfFileExists{xurl.sty}{\usepackage{xurl}}{} % add URL line breaks if available
\IfFileExists{bookmark.sty}{\usepackage{bookmark}}{\usepackage{hyperref}}
\hypersetup{
  pdftitle={arc42 Template},
  hidelinks,
  pdfcreator={LaTeX via pandoc}}
\urlstyle{same} % disable monospaced font for URLs
\usepackage{longtable,booktabs,array}
\usepackage{calc} % for calculating minipage widths
% Correct order of tables after \paragraph or \subparagraph
\usepackage{etoolbox}
\makeatletter
\patchcmd\longtable{\par}{\if@noskipsec\mbox{}\fi\par}{}{}
\makeatother
% Allow footnotes in longtable head/foot
\IfFileExists{footnotehyper.sty}{\usepackage{footnotehyper}}{\usepackage{footnote}}
\makesavenoteenv{longtable}
\usepackage{graphicx}
\makeatletter
\def\maxwidth{\ifdim\Gin@nat@width>\linewidth\linewidth\else\Gin@nat@width\fi}
\def\maxheight{\ifdim\Gin@nat@height>\textheight\textheight\else\Gin@nat@height\fi}
\makeatother
% Scale images if necessary, so that they will not overflow the page
% margins by default, and it is still possible to overwrite the defaults
% using explicit options in \includegraphics[width, height, ...]{}
\setkeys{Gin}{width=\maxwidth,height=\maxheight,keepaspectratio}
% Set default figure placement to htbp
\makeatletter
\def\fps@figure{htbp}
\makeatother
\setlength{\emergencystretch}{3em} % prevent overfull lines
\providecommand{\tightlist}{%
  \setlength{\itemsep}{0pt}\setlength{\parskip}{0pt}}
\setcounter{secnumdepth}{-\maxdimen} % remove section numbering
\ifLuaTeX
  \usepackage{selnolig}  % disable illegal ligatures
\fi

\usepackage[citestyle=numeric, sorting=none]{biblatex}


\addbibresource{Literatur.bib}

\title{\includegraphics{resources/header.png} Projektbericht Trashpong}
\author{}
\date{Januar 2023}

\begin{document}
\maketitle

% DOCUMENT STARTS HERE
\hypertarget{section-introduction-and-goals}{%
\section{Architektur}\label{section-introduction-and-goals}}

Im Rahmen der Architektur wird eine zentralisierte Architektur verwendet, wobei es sich um eine Client-Server-Architektur handelt (vgl. Abbildung \ref{fig:clientserver}).
Die Programmteile werden in zwei separate Einheiten unterteilt, um eine klare Trennung zu gewährleisten. Der Server stellt einen bestimmten Service bereit und leitet die Anfragen an die Clients weiter. Die Clients empfangen diese Anfragen.\cite{tanenbaum2007distributed}

Der Client ist eine kompilierte Executable, geschrieben in der Open-Source Godot Engine \cite{godot}. Dort befindet sich die Gesamte Spiel Logik über die der Benutzer das Spiel spielen kann. Der Server befindet sich in einem Docker Compose bestehend aus einer Posgres Datenbank, einem Load Balancer und beliebig skalierbare Spieleserver welche die Anfragen der Clients verarbeitet. 
\begin{figure}[H]
	\centering
	\includegraphics[width=\textwidth ]{resources/Client-Server.drawio.png}
	\caption{Kontext-Sicht der Client-Server-Architektur des Pong Spiels}
	\label{fig:clientserver}
\end{figure}
\subsection{Architektural Decision}
Erklärung, warum diese Architektur gewählt wurde (z. B. Skalierbarkeit, Modularität, Sicherheit).

\section{Systemkomponenten}

\subsection{Anforderungen}
\begin{center}
  \begin{tabular}{|p{\linewidth}|}
    \hline
    \textbf{Spieler-Registrierung und -Login (FA1)} \\
    Beschreibung: Benutzer müssen sich mit ihren Namen anmelden können, um am Spiel teilzunehmen.
    Die Nutzung sollte niederschwelling sein. Ein Benutzer muss deshalb kein Konto anlegenc \\ \\
    \hline
    \textbf{Echtzeit-Multiplayer-Funktionalität (FA2)} \\
    Beschreibung: Das Spiel muss in der Lage sein, mehrere Spieler in Echtzeit zu verbinden und ein synchronisiertes Spiel zu ermöglichen. Die Bewegungen der Schläger und der Ball müssen in Echtzeit zwischen den Spielern synchronisiert werden.\\ \\
    \hline
    \textbf{Punkteverwaltung (FA3)} \\
    Beschreibung: Das System muss die Punkte der Spieler während des Spiels erfassen und verwalten können. Nach jedem Spiel sollte ein Punktestand angezeigt werden, der den Gewinner ermittelt. \\ \\
    \hline
  \end{tabular}
\end{center}

\begin{center}
  \begin{tabular}{|p{\linewidth}|}
    \hline
    \textbf{Leistung und Skalierbarkeit (NF1)} \\
    Beschreibung: Das Spiel sollte auch bei hoher Benutzerzahl flüssig und ohne Verzögerungen laufen. Das System muss skalierbar sein, um eine große Anzahl von gleichzeitigen Spielern zu unterstützen. \\ \\
    \hline
    \textbf{Sicherheit (NF2)} \\
    Beschreibung: Die Benutzerdaten, einschließlich Anmeldedaten und Spielstatistiken, müssen sicher gespeichert und übertragen werden. Das System sollte gegen häufige Sicherheitsbedrohungen wie SQL-Injektionen und Cross-Site-Scripting geschützt sein.\\ \\
    \hline
    \textbf{Datensparsamkeit (NF3)} \\
     Beschreibung:
     Die Datenerhebung und -speicherung wird nur im notwendigen Maß durchgeführt. 
     Ziel ist es, nur die Daten zu erfassen und zu speichern, die für den Betrieb und die Funktionalität des Systems unbedingt erforderlich sind. 
     Dies trägt zum Schutz der Privatsphäre der Nutzer bei und reduziert das Risiko von Datenmissbrauch und -verlust.
     Deshalb sollen nur der Nutzername und die Punkte gespeichert werden.
    \\ \\
    \hline
    \textbf{Benutzerfreundlichkeit (NF4)} \\
    Beschreibung: Die Benutzeroberfläche des Spiels sollte intuitiv und leicht zu bedienen sein. Neue Spieler sollten sich schnell zurechtfinden und das Spiel ohne umfangreiche Anleitungen verstehen können.\\ \\
    \\ \\
    \hline
  \end{tabular}
\end{center}

\section{Umsetzung}

\subsection{Implementierung}
\begin{itemize}
    \item \textbf{Umsetzung der Architektur}: Beschreibung der Implementierung der einzelnen Komponenten und ihrer Interaktionen.
    \item \textbf{Schwierigkeiten und Lösungen}: Was waren die technischen Herausforderungen und wie wurden sie bewältigt?
\end{itemize}

\begin{figure}[H]
	\centering
	\includegraphics[width=\textwidth ]{resources/login.pdf}
	\caption{Dein TExt hier}
	\label{fig:ablaufdiagramm-login}
\end{figure}
\begin{figure}[H]
	\centering
	\includegraphics[width=\textwidth ]{resources/show_lobby.pdf}
	\caption{Dein TExt hier}
	\label{fig:ablaufdiagramm-show_lobby}
\end{figure}
\begin{figure}[H]
	\centering
	\includegraphics[width=\textwidth ]{resources/join_lobby.pdf}
	\caption{Dein TExt hier}
	\label{fig:ablaufdiagramm-join_lobby}
\end{figure}
\begin{figure}[H]
	\centering
	\includegraphics[width=\textwidth ]{resources/start_game.pdf}
	\caption{Dein TExt hier}
	\label{fig:ablaufdiagramm-start_game}
\end{figure}

\begin{figure}[H]
	\centering
	\includegraphics[width=\textwidth ]{resources/Event-Based-Spielablauf.png}
	\caption{Eventbasierter Ablauf des Spiels über einen Socket-IO Websocket. Es handelt sich um eine Vereinfachte Ansicht. Beide Seiten können die jeweiligen Events an den Server senden.}
	\label{fig:ablaufdiagramm-spiel}
\end{figure}

\subsection{Mögliche Alternativen}
\begin{itemize}
    \item \textbf{Architektur- und Technologiealternativen}: Überlegungen zu möglichen anderen Architekturen und Technologien sowie deren Vor- und Nachteile.
\end{itemize}

\section{Reflexion}

\subsection{Rückblick}
\begin{itemize}
    \item \textbf{Änderungen nach dem Projekt}: Was würde man im Nachhinein anders machen, um das System zu verbessern?
\end{itemize}

\subsection{Herausforderungen}
\begin{itemize}
    \item \textbf{Größte Herausforderungen}: Rückblick auf die bedeutendsten Schwierigkeiten und wie sie gelöst wurden.
\end{itemize}
\newpage
 \printbibliography[title={Quellen}]
\end{document}
